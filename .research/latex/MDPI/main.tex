%  LaTeX support: latex@mdpi.com
%  For support, please attach all files needed for compiling as well as the log file, and specify your operating system, LaTeX version, and LaTeX editor.

%=================================================================
\documentclass[journal,article,submit,pdftex,moreauthors]{Definitions/mdpi}

%=================================================================
% MDPI internal commands - do not modify
\firstpage{1}
\makeatletter
\setcounter{page}{\@firstpage}
\makeatother
\pubvolume{1}
\issuenum{1}
\articlenumber{0}
\pubyear{2026}
\copyrightyear{2026}
%\externaleditor{Firstname Lastname} % More than 1 editor, please add `` and '' before the last editor name
\datereceived{ }
\daterevised{ } % Comment out if no revised date
\dateaccepted{ }
\datepublished{ }
%\datecorrected{} % For corrected papers: "Corrected: XXX" date in the original paper.
%\dateretracted{} % For retracted papers: "Retracted: XXX" date in the original paper.
%\doinum{} % Used for some special journals, like molbank
%\pdfoutput=1 % Uncommented for upload to arXiv.org
%\CorrStatement{yes}  % For updates
%\longauthorlist{yes} % For many authors that exceed the left citation part
%\IsAssociation{yes} % For association journals

%=================================================================
% Add packages and commands here. The following packages are loaded in our class file: fontenc, inputenc, calc, indentfirst, fancyhdr, graphicx, epstopdf, lastpage, ifthen, float, amsmath, amssymb, lineno, setspace, enumitem, mathpazo, booktabs, titlesec, etoolbox, tabto, xcolor, colortbl, soul, multirow, microtype, tikz, totcount, changepage, attrib, upgreek, array, tabularx, pbox, ragged2e, tocloft, marginnote, marginfix, enotez, amsthm, natbib, hyperref, cleveref, scrextend, url, geometry, newfloat, caption, draftwatermark, seqsplit
% cleveref: load \crefname definitions after \begin{document}

%=================================================================
% Please use the following mathematics environments: Theorem, Lemma, Corollary, Proposition, Characterization, Property, Problem, Example, ExamplesandDefinitions, Hypothesis, Remark, Definition, Notation, Assumption
%% For proofs, please use the proof environment (the amsthm package is loaded by the MDPI class).

%=================================================================
% Full title of the paper (Capitalized)
\Title{CFS-eSAFE\@: Cross-Fitted Fixed-Sequence E-Value Gatekeeping for Safer Iterative Prompt Optimization}

% Author Orchid ID: enter ID or remove command
\newcommand{\orcidauthorA}{0000{-}0000{-}0000{-}000X} % Add \orcidA{} behind the author's name
%\newcommand{\orcidauthorB}{0000-0000-0000-000X} % Add \orcidB{} behind the author's name

% Authors, for the paper (add full first names)
\Author{Firstname Lastname $^{1}$\orcidA{}, Firstname Lastname $^{2}$ and Firstname Lastname $^{2,}$*}

%\longauthorlist{yes}

% MDPI internal command: Authors, for metadata in PDF
\AuthorNames{Firstname Lastname, Firstname Lastname and Firstname Lastname}

% Affiliations / Addresses (Add [1] after \address if there is only one affiliation.)
\address{%
$^{1}$ \quad Affiliation 1; e-mail@e-mail.com\\
$^{2}$ \quad Affiliation 2; e-mail@e-mail.com}

% Contact information of the corresponding author
\corres{Correspondence: e-mail@e-mail.com; Tel.:\ (optional; include country code; if there are multiple corresponding authors, add author initials) +xx-xxxx-xxx-xxxx (F.L.)}

% Current address and/or shared authorship
%\firstnote{Current address: Affiliation.}
% Current address should not be the same as any items in the Affiliation section.

%\secondnote{These authors contributed equally to this work.}
% The commands \thirdnote{} till \eighthnote{} are available for further notes.

%\simplesumm{} % Simple summary

%\conference{} % An extended version of a conference paper

% Abstract (Do not insert blank lines, i.e. \\)
\abstract{Iterative prompt editing can improve a language model's behavior without weight updates, but deployment often requires evidence that each accepted update does not introduce new failures on inputs that matter most. This is difficult because two forms of adaptivity are unavoidable: candidate prompts are generated after inspecting failures, and engineers then discover multiple high-dimensional ``stress slices'' only after observing behavior. Naively certifying many slices with multiplicity corrections across candidates, iterations, and slices can destroy statistical power under small audits. We propose CFS-eSAFE, a training-free prompt optimizer that enforces strict data-role separation via cross-fitting, mines and orders an explicit list of the most regressive discovered slices, and certifies non-regression using anytime-valid mixture e-values in a fixed-sequence (gatekeeping) design, combined with alpha-wealth to control risk across repeated candidate tests. We implement an open-model pipeline for constrained sentiment-reversal rewriting and compare to a Bonferroni-gated e-value baseline. In the executed runs, however, the optimizer evaluated zero candidate prompts and achieved zero constrained accuracy across domains for both methods, preventing empirical validation. We report these outcomes transparently, analyze what parts of the pipeline were not exercised, and identify concrete prerequisites required for future evaluations of the proposed safety contract.}

% Keywords
\keyword{keyword 1; keyword 2; keyword 3 (List three to ten pertinent keywords specific to the article; yet reasonably common within the subject discipline.)}

% The fields PACS, MSC, and JEL may be left empty or commented out if not applicable
%\PACS{J0101}
%\MSC{}
%\JEL{}

%%%%%%%%%%%%%%%%%%%%%%%%%%%%%%%%%%%%%%%%%%
% Only for the journal Diversity
%\LSID{\url{http://}}

%%%%%%%%%%%%%%%%%%%%%%%%%%%%%%%%%%%%%%%%%%
% Only for the journal Applied Sciences
%\featuredapplication{Authors are encouraged to provide a concise description of the specific application or a potential application of the work. This section is not mandatory.}
%%%%%%%%%%%%%%%%%%%%%%%%%%%%%%%%%%%%%%%%%%

%%%%%%%%%%%%%%%%%%%%%%%%%%%%%%%%%%%%%%%%%%
% Only for the journal Data
%\dataset{DOI number or link to the deposited data set if the data set is published separately. If the data set shall be published as a supplement to this paper, this field will be filled by the journal editors. In this case, please submit the data set as a supplement.}
%\datasetlicense{License under which the data set is made available (CC0, CC-BY, CC-BY-SA, CC-BY-NC, etc.)}

%%%%%%%%%%%%%%%%%%%%%%%%%%%%%%%%%%%%%%%%%%
% Only for the journal BioTech, Fishes, Neuroimaging and Toxins
%\keycontribution{The breakthroughs or highlights of the manuscript. Authors can write one or two sentences to describe the most important part of the paper.}

%%%%%%%%%%%%%%%%%%%%%%%%%%%%%%%%%%%%%%%%%%
% Only for the journal Encyclopedia
%\encyclopediadef{For entry manuscripts only: please provide a brief overview of the entry title instead of an abstract.}

%%%%%%%%%%%%%%%%%%%%%%%%%%%%%%%%%%%%%%%%%%
% Different journals have different requirements. Please check the specific journal guidelines in the "Instructions for Authors" on the journal's official website.
%\addhighlights{yes}
%\renewcommand{\addhighlights}{%
%
%\noindent The goal is to increase the discoverability and readability of the article via search engines and other scholars. Highlights should not be a copy of the abstract, but a simple text allowing the reader to quickly and simplified find out what the article is about and what can be cited from it. Each of these parts should be devoted up to 2~bullet points.\vspace{3pt}\\
%\textbf{What are the main findings?}
% \begin{itemize}[labelsep=2.5mm,topsep=-3pt]
% \item First bullet.
% \item Second bullet.
% \end{itemize}\vspace{3pt}
%\textbf{What are the implications of the main findings?}
% \begin{itemize}[labelsep=2.5mm,topsep=-3pt]
% \item First bullet.
% \item Second bullet.
% \end{itemize}
%}

%%%%%%%%%%%%%%%%%%%%%%%%%%%%%%%%%%%%%%%%%%
\begin{document}

%%%%%%%%%%%%%%%%%%%%%%%%%%%%%%%%%%%%%%%%%%
\section{Introduction}

Prompt-based test-time optimization seeks to improve a model's behavior without changing its parameters, enabling rapid iteration and avoiding training instabilities. A representative approach is SELF-REFINE, which alternates between generating feedback and revising an output using the same model, demonstrating gains on diverse tasks including sentiment reversal~\cite{madaan-2023-self}. While such methods are attractive for practical iteration, many deployments require more than improved average performance: they require a non-regression contract that rules out harmful degradations on inputs that are operationally important, safety-critical, or likely to be scrutinized.

This paper studies a safety gap that becomes acute when iterative prompt optimization is deployed in realistic engineering workflows. Two adaptivity sources interact in a way that breaks common evaluation and gating patterns.

First, stress-slice discovery is inevitable and high-dimensional. After observing a model's behavior, practitioners (or automated slice-mining tools) discover multiple failure pockets defined by combinations of simple features such as input length, the presence of negation, entity density, or the presence of socially salient identity terms. Importantly, regressions need not concentrate in a single ``worst'' slice: a prompt update can improve one region while degrading another, and multiple distinct pockets can co-exist.

Second, multiplicity becomes power-killing when many slices and many candidates are tested. Iterative prompt optimization evaluates a stream of candidate prompts that are themselves chosen after inspecting failures. If each candidate must be certified against a large family of slices using conservative multiplicity corrections (for example, Bonferroni across slices, across candidate attempts, and across iterations), then under a small audit budget it can become statistically infeasible to accept any update. This outcome encourages heuristic acceptance and undermines end-to-end safety.

We ask a deployment-motivated question: can we search adaptively over candidate prompt edits while maintaining an anytime-valid non-regression contract that extends beyond an overall audit average to an adaptively discovered list of stress slices, without collapsing power as the slice list grows?

We propose CFS-eSAFE (Cross-Fitted Fixed-Sequence e-value Safety), a discover-order-certify pipeline that composes three ideas into an end-to-end gate:

\begin{itemize}
\item \textbf{Cross-fitted data roles:} hold out an audit set that is never used to propose edits, and split that audit set into a discovery portion used to mine and order slices and a confirmation portion used only for statistical certification.
\item \textbf{Anytime-valid mixture e-values on paired differences:} compare a candidate prompt to the current prompt on the same examples using paired differences, and certify improvement or non-regression with mixture e-values that remain valid under optional stopping and repeated testing.
\item \textbf{Fixed-sequence (gatekeeping) certification over an ordered slice list:} rather than testing all slices simultaneously with a multiplicity split, test the discovered slices in a fixed order, requiring each gate to pass before proceeding, thereby avoiding spending error budget on slices that are never reached.
\end{itemize}

We verify the proposal by implementing the pipeline with open models on a constrained sentiment-reversal rewriting task. We also implement a baseline, AV-DR-SPO, that uses the same mixture e-value machinery but handles slice multiplicity via Bonferroni-style thresholds over a fixed slice family.

In the executed runs, both methods produce degenerate outcomes: constrained accuracy is $0.0$ across all evaluation domains, and the optimizer evaluates zero candidate prompts and accepts zero updates. While these results prevent an empirical test of the intended power and safety claims, they are still informative. They indicate that, in this particular instantiation, the task and constraint calibration yields almost no successes and the candidate proposal/evaluation loop terminates before the statistical gates are exercised. Reporting these outcomes is important because it highlights prerequisites for meaningful empirical study of safety gating in training-free prompt optimization.

\textbf{Contributions:}
\begin{itemize}
\item \textbf{Adaptive prompt-optimization setting:} We formalize a deployment-oriented prompt optimization setting in which both candidate prompts and the set of audited stress slices are adaptively determined.
\item \textbf{CFS-eSAFE gate design:} We introduce CFS-eSAFE, combining cross-fitted slice discovery and ordering, paired-difference evaluation, anytime-valid mixture e-values, fixed-sequence slice gatekeeping, and alpha-wealth to budget risk across an adaptive candidate search.
\item \textbf{Regression certificates:} We define an interpretable regression-certificate interface that reports which safety gate fails first (optimization gate, overall audit gate, or a specific discovered slice), with the intent of making rejections actionable.
\item \textbf{Transparent end-to-end implementation:} We implement an end-to-end open-model evaluation pipeline, report the executed results in full (including all generated figures), and analyze why the runs are degenerate and what must be fixed to test the core hypothesis.
\end{itemize}

Future work is primarily empirical and engineering-oriented. The constrained success definition must be calibrated so that baseline performance is non-zero; otherwise no gating method can demonstrate a safety-power tradeoff. In addition, candidate generation must reliably produce and evaluate candidate prompts so that the gatekeeping design and certificates are exercised. Once the pipeline is non-degenerate, the central scientific comparison becomes testable: whether fixed-sequence e-value gatekeeping can certify a list of adaptively discovered slices with materially higher acceptance power than Bonferroni while maintaining a valid non-regression contract under adaptive candidate testing and optional stopping.

%%%%%%%%%%%%%%%%%%%%%%%%%%%%%%%%%%%%%%%%%%
\section{Related Work}

Our work is most directly connected to training-free, test-time iterative refinement and prompt-editing methods that use the model itself to critique and improve outputs. SELF-REFINE alternates between generating natural-language feedback and producing a refined output, reporting improvements across diverse tasks without additional training, reinforcement learning, or external reward models~\cite{madaan-2023-self}. CFS-eSAFE shares the training-free motivation and operates in a similar ``propose then refine'' loop, but targets a different research problem: statistically valid deployment gating under adaptivity.

A key contrast is that SELF-REFINE is primarily an optimization procedure. It aims to produce better outputs (often by iterating several times per input) and may use task-specific selection heuristics to choose among iterations~\cite{madaan-2023-self}. CFS-eSAFE is instead a safety and evaluation layer over an optimizer: it decides whether a prompt update should be accepted for future use based on evidence about non-regression. In deployment, this distinction matters because a prompt update can improve average behavior while degrading concentrated pockets of inputs that are rare overall but important for robustness auditing.

Our work also contrasts with slice testing practices that certify only (i) an overall audit average, (ii) a small predeclared list of slices, or (iii) at most a single mined ``worst slice''. These practices can be operationally convenient, but they do not address the realistic case where multiple distinct regressions can co-exist. CFS-eSAFE explicitly aims to certify a list of slices that are adaptively discovered from observed behavior. Because the slice list is chosen based on data, the method must prevent ``double dipping'' where the same audit examples are used both to select slices and to certify them. CFS-eSAFE's cross-fitted discovery/confirmation split is designed to address this selection leakage.

Finally, our work contrasts with conservative multiplicity control when many slices are audited. A natural safe-prompt-optimization baseline is to use valid sequential evidence (such as e-values) but apply Bonferroni-style corrections across slices, and possibly across candidate attempts. This can become overly conservative under small audits because the per-slice thresholds become very hard to pass as the number of tested slices grows. Our baseline AV-DR-SPO instantiates this approach: it uses mixture e-values but corrects slice multiplicity via Bonferroni-style thresholds over a fixed slice set. CFS-eSAFE differs by using a fixed-sequence (gatekeeping) design over an ordered list: later slice tests are only performed after earlier tests pass. In effect, it avoids allocating error budget to slices that would never be reached due to earlier failures.

In summary, relative to SELF-REFINE and related training-free refinement approaches~\cite{madaan-2023-self}, the novelty of CFS-eSAFE is not in proposing candidates but in certifying them under adaptive candidate generation and adaptive slice discovery: cross-fitted slice mining, statistically valid mixture e-values, fixed-sequence gatekeeping over multiple slices, and interpretable rejection certificates intended to support debugging and auditing.

%%%%%%%%%%%%%%%%%%%%%%%%%%%%%%%%%%%%%%%%%%
\section{Background}

We consider a training-free prompt optimization loop for a fixed rewriting model. The goal is to update the instruction prompt while maintaining a deployment-oriented non-regression contract under adaptive candidate testing and adaptive stress-slice discovery.

\subsection{Problem setting and constrained success}
Let $x$ denote an input text and $y$ its gold sentiment label in $\{0,1\}$. A rewriting model produces a rewrite $r(P,x)$ given a prompt $P$. We evaluate rewrites using a constrained success indicator $z(P,x)\in\{0,1\}$ that requires three conditions simultaneously:

\begin{enumerate}
\item Sentiment reversal success: a fixed sentiment classifier applied to $r(P,x)$ predicts the opposite of $y$.
\item Semantic preservation: cosine similarity between embeddings of $x$ and $r(P,x)$ is at least a threshold $\tau$.
\item Fluency: a language model assigns $r(P,x)$ a negative log-likelihood (NLL) not exceeding $\mathrm{nll}_{\max}$. In the implementation, this is sometimes described equivalently as perplexity not exceeding $\mathrm{ppl}_{\max}$ where $\mathrm{ppl}_{\max}=\exp(\mathrm{nll}_{\max})$.
\end{enumerate}

\subsection{Paired differences for candidate comparison}
At each step, we compare a candidate prompt $P_c$ to the current prompt $P$ on the same examples. For an example $x_i$, define the paired difference $d_i = z(P_c,x_i)-z(P,x_i)$. Because $z$ is binary, $d_i\in\{-1,0,1\}$. Paired differences provide variance reduction because each example serves as its own control; this is especially important when audits are small.

\subsection{Stress slices and slice keys}
To capture concentrated failure pockets, we define a deterministic slicing function $s(x)$ that maps each input to a discrete slice key. In our implementation, the slice key is the conjunction of four text-derived features:
\begin{itemize}
\item a length bin (short if length is at most $8$ tokens, medium if $9$-$18$ tokens, long otherwise),
\item a negation indicator based on the presence of ``not'' or ``n't'',
\item an entity-density proxy based on the fraction of tokens whose first character is uppercase, thresholded at $0.18$,
\item an identity-term flag indicating whether the text contains one of the terms $\{\text{muslim},\text{black},\text{gay},\text{disabled}\}$.
\end{itemize}
A slice is the subset of examples sharing the same slice key. The intended safety requirement is that a candidate prompt be non-regressive not only overall, but also on an adaptively discovered list of high-risk slices.

\subsection{Anytime-valid evidence via e-values}
We need tests that remain valid under optional stopping and repeated use across many candidate prompts. We use e-values: nonnegative random variables $E$ such that, under the null hypothesis, $\mathbb{E}[E]\le 1$. A common decision rule is to reject a null hypothesis at level $\ell$ when $E\ge 1/\ell$; validity is preserved under optional stopping.

We focus on one-sided hypotheses about the mean paired difference. Each gate tests a null of the form $H_0:\mathbb{E}[d]\le m$, for a margin $m$. In our application, $m=0$ is used to require improvement (no worse on average), while $m=-\delta$ is used to allow a small tolerated degradation $\delta$ in a non-regression gate.

\subsection{Mixture e-values for bounded differences}
Because $d$ is bounded in $[-1,1]$, we can construct multiplicative e-values using betting fractions. For a fixed betting fraction $\lambda$, define
\[
E_{\lambda} = \prod_{i=1}^{n}\bigl(1+\lambda\,(d_i-m)\bigr).
\]
Using multiple $\lambda$ values can improve robustness, but selecting $\lambda$ adaptively based on the same data can invalidate the e-value. We therefore fix a small grid of $\lambda$ values and use a mixture:
\[
E = \frac{1}{|\Lambda|}\sum_{\lambda\in\Lambda} E_{\lambda}.
\]
This mixture remains a valid e-value, whereas taking the maximum over $\lambda$ is generally not guaranteed to preserve validity. This detail matters because the overall optimizer tests many candidates and may stop early based on the observed evidence.

%%%%%%%%%%%%%%%%%%%%%%%%%%%%%%%%%%%%%%%%%%
\section{Method}

CFS-eSAFE (Cross-Fitted Fixed-Sequence e-value Safety) is a safety-gated, training-free prompt optimization procedure. It augments a standard iterative prompt-editing loop with a certification mechanism intended to remain valid under two adaptivity sources: candidate prompts are generated after observing failures, and stress slices are mined after observing behavior.

\subsection{Data roles and cross-fitting}
For each random seed, data are partitioned into three disjoint roles:
\begin{itemize}
\item $D_{\mathrm{opt}}$: an optimization batch used to identify failures and to guide candidate prompt proposals. $D_{\mathrm{opt}}$ provides no safety guarantee.
\item $A$: an audit set that is never used to propose edits. $A$ is split into $A_{\mathrm{disc}}$ for slice discovery and ordering and $A_{\mathrm{conf}}$ for statistical confirmation.
\end{itemize}
This split is fixed once per seed. For each candidate prompt $P_c$, slices and their order are determined using only $A_{\mathrm{disc}}$, and the resulting ordered slice list is treated as fixed when testing on $A_{\mathrm{conf}}$. This cross-fitting aims to prevent slice-selection leakage: the evidence used to choose which slices to test is not reused to pass those slice tests.

\subsection{Gate statistics: paired differences and mixture e-values}
For each example $x$, we compute $z(P,x)$ and $z(P_c,x)$ and then form $d=z(P_c,x)-z(P,x)$. All gates operate on collections of these differences, either overall or restricted to a slice key $s(x)$.

For a list of differences ${\{d_i\}}_{i=1}^{n}$ and a margin $m$, we compute a mixture e-value. Fix a grid $\Lambda=\{0.05,0.1,0.2,0.4,0.7\}$. For each $\lambda\in\Lambda$, compute
\[
E_{\lambda}=\prod_{i=1}^{n} \max\{\epsilon,\,1+\lambda\,(d_i-m)\},
\]
where $\epsilon>0$ is a small constant used to clip factors for numerical stability. The mixture e-value is
\[
E = \frac{1}{|\Lambda|}\sum_{\lambda\in\Lambda} E_{\lambda}.
\]
A gate at level $\ell$ passes if $E\ge 1/\ell$.

\subsection{Adaptive slice mining and fixed-sequence certification}
\textbf{Slice mining and ordering.} For each candidate prompt $P_c$, we compute paired differences on $A_{\mathrm{disc}}$ and aggregate them by slice key. Among slices with support at least $n_{\min}$, we compute the mean paired difference per slice and select the top $J$ most regressive slices (most negative means). We then order this list from most to least regressive, yielding an ordered slice list $(s_1,\dots,s_J)$ that is deterministic given $(P,P_c,A_{\mathrm{disc}})$.

\textbf{Fixed-sequence (gatekeeping) certification.} For each candidate, CFS-eSAFE runs a fixed sequence of gates, each tested at the same level $\ell$:
\begin{enumerate}
\item Optimization improvement gate on $D_{\mathrm{opt}}$: test $H_{0,\mathrm{opt}}:\mathbb{E}[d]\le 0$.
\item Overall audit non-regression gate on $A_{\mathrm{conf}}$: test $H_{0,\mathrm{all}}:\mathbb{E}[d]\le -\delta$.
\item Slice-list non-regression gates on $A_{\mathrm{conf}}$ restricted to each discovered slice $s_j$ in the $A_{\mathrm{disc}}$ order: test $H_{0,s_j}:\mathbb{E}[d\mid s(x)=s_j]\le -\delta$.
\end{enumerate}
The acceptance rule is conjunctive: accept $P_c$ only if every gate in the sequence passes. The fixed-sequence design is motivated by power preservation when $J$ is large: unlike Bonferroni, the method does not divide $\ell$ by $J$.

\textbf{Alpha-wealth across repeated candidate tests.} Across an adaptive search over many candidates, CFS-eSAFE maintains alpha-wealth $W$ initialized at $\alpha$. For each candidate test, it spends $\mathrm{spend}=\min(\alpha_{\max\,\mathrm{spend}}, W/4)$, sets $\ell=\mathrm{spend}$ for the gates, and uses the threshold $1/\ell$. If the candidate is rejected, wealth decreases by $\mathrm{spend}$. If the candidate is accepted, wealth is replenished by $\rho\,\mathrm{spend}$ (with a cap in the implementation), where $\rho\in{(0,1]}$ is a reward fraction.

\textbf{Regression certificates.} For each rejected candidate prompt, CFS-eSAFE records the first gate that failed, chosen from $\{\text{opt},\text{audit\_all},\text{slice:}\langle\text{slice\_key}\rangle\}$. This interface is intended to produce actionable explanations for rejections.

\begin{algorithm}[H]
\caption{CFS-eSAFE safety-gated prompt optimization (single seed)}
\label{alg:cfs_esafe}
\begin{algorithmic}[1]
\State Partition data into $D_{\mathrm{opt}}$, $A_{\mathrm{disc}}$, and $A_{\mathrm{conf}}$
\State Initialize prompt $P\leftarrow P_{0}$ and wealth $W\leftarrow \alpha$
\For{$t=1$ to $T$}
    \State Propose a set of candidate prompts $\mathcal{C}_t$ using only $D_{\mathrm{opt}}$
    \ForAll{$P_c\in\mathcal{C}_t$}
        \If{$W \le 0$}
            \State \textbf{break}
        \EndIf
        \State $\mathrm{spend} \leftarrow \min(\alpha_{\max\,\mathrm{spend}}, W/4)$; $\ell \leftarrow \mathrm{spend}$
        \State Compute paired differences on $A_{\mathrm{disc}}$; mine ordered slices $(s_1,\dots,s_J)$
        \State Compute $E_{\mathrm{opt}}$ from paired differences on $D_{\mathrm{opt}}$ with margin $m=0$
        \If{$E_{\mathrm{opt}} < 1/\ell$}
            \State Record certificate $\text{fail}=\text{opt}$
            \State $W \leftarrow W-\mathrm{spend}$; \textbf{continue}
        \EndIf
        \State Compute $E_{\mathrm{all}}$ on $A_{\mathrm{conf}}$ with margin $m=-\delta$
        \If{$E_{\mathrm{all}} < 1/\ell$}
            \State Record certificate $\text{fail}=\text{audit\_all}$
            \State $W \leftarrow W-\mathrm{spend}$; \textbf{continue}
        \EndIf
        \For{$j=1$ to $J$}
            \State Compute $E_{s_j}$ on $\{x\in A_{\mathrm{conf}}: s(x)=s_j\}$ with margin $m=-\delta$
            \If{$E_{s_j} < 1/\ell$}
                \State Record certificate $\text{fail}=\text{slice:}s_j$
                \State $W \leftarrow W-\mathrm{spend}$; \textbf{break}
            \EndIf
        \EndFor
        \If{all gates passed}
            \State Accept update: $P\leftarrow P_c$
            \State $W \leftarrow \min(W_{\max},\, W-\mathrm{spend}+\rho\,\mathrm{spend})$
        \EndIf
    \EndFor
\EndFor
\end{algorithmic}
\end{algorithm}

%%%%%%%%%%%%%%%%%%%%%%%%%%%%%%%%%%%%%%%%%%
\section{Experimental Setup}

\subsection{Task, goal, and models}
We evaluate training-free prompt optimization on constrained sentiment-reversal rewriting. Given an input text $x$ with sentiment label $y$, the rewriting model must produce a fluent rewrite whose sentiment is the opposite of $y$ while preserving semantic content.

The rewriting model and the prompt editor are the same open model: \texttt{google/flan-t5-small}. Sentiment reversal is assessed using \texttt{distilbert-base-uncased-finetuned-sst-2-english}. Semantic similarity is computed using \texttt{sentence-transformers/all-MiniLM-L6-v2} with cosine similarity. Fluency is measured using GPT-2 negative log-likelihood (with optional conversion to perplexity).

\subsection{Datasets and splits}
The optimization dataset $D_{\mathrm{opt}}$ is a subset of SST-2 train with $120$ examples. The audit pool (never used to propose edits) is formed from:
\begin{itemize}
\item Yelp train: $80$ examples,
\item IMDb train: $80$ examples,
\item SST-2 probe augmentations: $80$ examples created by deterministic perturbations (truncation and appended negation),
\item Identity-term probe augmentations: $40$ examples created by inserting a neutral identity prefix such as ``As a Muslim person, \dots''.
\end{itemize}
From this pool, we sample an audit set $A$ of size $200$ and split it once into $A_{\mathrm{disc}}$ ($100$) and $A_{\mathrm{conf}}$ ($100$).

Final evaluation is performed on SST-2 validation ($300$ examples), Yelp test ($300$ examples), and IMDb test ($300$ examples). The evaluation code also constructs held-out probes from SST-2 validation by applying the same deterministic negation and identity insertions; these probes are used when computing worst-slice constrained accuracy.

\subsection{Constrained success definition and thresholds}
For a prompt $P$ and example $(x,y)$, we generate $r(P,x)$ with the rewriter. We set $z(P,x)=1$ if all three conditions hold:
\begin{itemize}
\item the sentiment classifier predicts the opposite label,
\item cosine similarity is at least $\tau_{\mathrm{similarity}}$,
\item GPT-2 NLL is at most $\mathrm{nll}_{\max}$.
\end{itemize}
In the configuration system, $\mathrm{nll}_{\max}$ is represented as \texttt{pi\_fluency\_nll}, and $\mathrm{ppl}_{\max}$ is computed as $\exp(\mathrm{nll}_{\max})$. Across executed runs, Optuna-selected values include (as recorded in summaries) $\tau_{\mathrm{similarity}}\approx 0.691$ with $\mathrm{nll}_{\max}\approx 2.487$ (so $\mathrm{ppl}_{\max}\approx 12.03$) in one run, and $\tau_{\mathrm{similarity}}\approx 0.728$ with $\mathrm{nll}_{\max}\approx 3.652$ (so $\mathrm{ppl}_{\max}\approx 38.53$) in another.

\subsection{Metrics, optimization protocol, and compared methods}
We report per-domain constrained accuracy for each of the three evaluation domains (SST-2, Yelp, IMDb). The primary metric used in the executed code is robust constrained accuracy defined as the minimum of the three per-domain constrained accuracies; i.e., $\min\{\mathrm{Acc}_{\mathrm{SST2}},\mathrm{Acc}_{\mathrm{Yelp}},\mathrm{Acc}_{\mathrm{IMDb}}\}$. This detail matters because it differs from a two-domain minimum described elsewhere in the surrounding documentation; in the executed runs and logged summaries, the robust objective is $\min(\mathrm{SST2},\mathrm{Yelp},\mathrm{IMDb})$.

We also report worst-slice constrained accuracy on a combined evaluation set (evaluation domains plus held-out probes). This metric is defined as the minimum slice-wise constrained accuracy across slice keys with support at least \texttt{slice\_min\_support}. Finally, an OOD regression indicator is computed per seed by comparing the method's Yelp or IMDb constrained accuracy to the fixed base prompt and flagging a regression if the drop exceeds $0.01$.

We compare the proposed method CFS-eSAFE to a baseline AV-DR-SPO that uses mixture e-values but applies Bonferroni-style multiplicity handling over a fixed family of slice keys during certification. Four runs are reported:
\begin{itemize}
\item \texttt{proposed-flan-t5-small-imdb} (CFS-eSAFE),
\item \texttt{comparative-1-flan-t5-small-imdb} (AV-DR-SPO),
\item \texttt{proposed-distilbert-sst2-imdb} (CFS-eSAFE),
\item \texttt{comparative-1-distilbert-sst2-imdb} (AV-DR-SPO).
\end{itemize}
The ``distilbert'' runs set a configuration field indicating a sentiment-evaluator role, but the rewriting model remains \texttt{google/flan-t5-small} across runs.

The prompt optimization loop allows up to $3$ iterations with up to $4$ candidate prompts per iteration. Slice mining uses \texttt{slice\_min\_support}$=12$. The proposed method mines a discovered list of top-$J$ slices (\texttt{J\_top\_slices} is tuned), while the baseline tests a fixed number of slices (\texttt{num\_slices\_tested} is tuned) and applies a Bonferroni-style threshold controlled by \texttt{bonferroni\_family}. Global alpha-wealth is initialized at $\alpha=0.1$, and per-candidate spend is bounded by \texttt{alpha\_max\_spend}. The base prompt is: ``Rewrite the text to the opposite sentiment while preserving meaning, entities, and topic. Write fluent English.''

Optuna is enabled with $20$ trials per run. For CFS-eSAFE, the search space includes $\tau_{\mathrm{similarity}}$, \texttt{pi\_fluency\_nll} (which sets $\mathrm{nll}_{\max}$ and thus $\mathrm{ppl}_{\max}$), \texttt{delta\_noninferiority}, \texttt{J\_top\_slices}, and \texttt{alpha\_max\_spend}. For the Bonferroni baseline, the search space includes \texttt{num\_slices\_tested} and \texttt{bonferroni\_family}. The selected hyperparameters are recorded in run summaries, and the evaluation script logs summary statistics for robust accuracy, worst-slice accuracy, OOD regression rate, accepted steps, and candidates evaluated.

%%%%%%%%%%%%%%%%%%%%%%%%%%%%%%%%%%%%%%%%%%
\section{Results}

This section reports results from the four executed runs and includes all figures produced by the evaluation script. The outcomes are degenerate across methods: all constrained accuracies are zero, and the optimization loop evaluates zero candidate prompts. We therefore focus on (i) documenting the observed metrics precisely and (ii) identifying which parts of the intended method were not exercised.

\subsection{Final constrained accuracy and robust constrained accuracy}
Across all four runs, the final per-domain constrained accuracies are $0.0$ on SST-2, Yelp, and IMDb. Consequently, the robust constrained accuracy (defined in code as the minimum over these three domains) is $0.0$ for every run. Reported standard deviations across seeds are $0.0$.

The cross-run aggregation reports best proposed and best baseline robust constrained accuracy values of $0.0$ and reports the relative gap as $\mathrm{NaN}$ because the baseline value is $0.0$. Worst-slice constrained accuracy is also $0.0$ (mean $0.0$, standard deviation $0.0$) for all runs. These aggregate outcomes are consistent with the per-domain results: if each domain accuracy is $0.0$, then any robust minimum and any worst-slice minimum must also be $0.0$.

\subsection{Optimization dynamics and acceptance power}
For all runs, the summary metrics \texttt{train/candidates\_evaluated\_mean}$=0.0$ and \texttt{train/accepted\_steps\_mean}$=0.0$. This implies that the optimization loop terminated without evaluating any candidate prompts, and therefore without running the proposed fixed-sequence slice gatekeeping or the baseline Bonferroni slice correction. As a result, the intended distinction between the methods (power preservation under many slices and many adaptive candidate tests) is not observable in these logs.

\subsection{Confusion-matrix diagnostics}
Confusion matrices logged for each domain are identical across methods and runs. On IMDb, the confusion counts are $\mathrm{tn}=726$, $\mathrm{fp}=0$, $\mathrm{fn}=774$, $\mathrm{tp}=0$. On SST-2, $\mathrm{tn}=758$, $\mathrm{fp}=7$, $\mathrm{fn}=728$, $\mathrm{tp}=7$. On Yelp, $\mathrm{tn}=766$, $\mathrm{fp}=1$, $\mathrm{fn}=732$, $\mathrm{tp}=1$.

Even where the sentiment target is sometimes met (nonzero $\mathrm{tp}$ on SST-2 and Yelp), constrained accuracy remains $0.0$. Since constrained success requires sentiment reversal, similarity, and fluency simultaneously, these results imply that at least one of the other constraints (semantic similarity threshold or fluency threshold) fails essentially everywhere under the executed configuration.

\subsection{Interpretation and limitations revealed by the executed runs}
The results indicate that two prerequisite conditions for empirical validation were not met in this instantiation. First, the constrained-success definition appears unattainable at a nontrivial rate under the selected thresholds in at least one run, notably when the fluency threshold is extremely strict (for example, $\mathrm{ppl}_{\max}\approx 12$ as implied by $\mathrm{nll}_{\max}\approx 2.487$). Second, the optimizer did not evaluate any candidates, so the statistical gates, slice discovery and ordering, and regression-certificate logging were not exercised.

Because the optimizer evaluated zero candidates, we cannot empirically compare acceptance power between fixed-sequence gatekeeping and Bonferroni correction, nor can we report a distribution of first-failed gates. The figures below should therefore be read as faithful records of the logged outputs from the executed runs rather than as evidence supporting (or refuting) the method's intended advantages.

\subsection{Figures}
\begin{figure}[H]
\centering
\includegraphics[width=0.7\linewidth]{images/comparative-1-distilbert-sst2-imdb_confusion_imdb.pdf}
\caption{Confusion matrix on IMDb for the Bonferroni baseline run (distilbert-labeled configuration); higher values on the diagonal indicate better performance.}
\label{fig:bonferroni_distilbert_confusion_imdb}
\end{figure}

\begin{figure}[H]
\centering
\includegraphics[width=0.7\linewidth]{images/comparative-1-distilbert-sst2-imdb_confusion_sst2.pdf}
\caption{Confusion matrix on SST-2 for the Bonferroni baseline run (distilbert-labeled configuration); higher values on the diagonal indicate better performance.}
\label{fig:bonferroni_distilbert_confusion_sst2}
\end{figure}

\begin{figure}[H]
\centering
\includegraphics[width=0.7\linewidth]{images/comparative-1-distilbert-sst2-imdb_confusion_yelp.pdf}
\caption{Confusion matrix on Yelp for the Bonferroni baseline run (distilbert-labeled configuration); higher values on the diagonal indicate better performance.}
\label{fig:bonferroni_distilbert_confusion_yelp}
\end{figure}

\begin{figure}[H]
\centering
\includegraphics[width=0.7\linewidth]{images/comparative-1-distilbert-sst2-imdb_domain_accuracy.pdf}
\caption{Per-domain constrained accuracy summary for the Bonferroni baseline run (distilbert-labeled configuration); higher values indicate better performance.}
\label{fig:bonferroni_distilbert_domain_accuracy}
\end{figure}

\begin{figure}[H]
\centering
\includegraphics[width=0.7\linewidth]{images/comparative-1-distilbert-sst2-imdb_learning_curve_robust.pdf}
\caption{Robust constrained accuracy over optimization steps for the Bonferroni baseline run (distilbert-labeled configuration); higher values indicate better performance.}
\label{fig:bonferroni_distilbert_learning_curve_robust}
\end{figure}

\begin{figure}[H]
\centering
\includegraphics[width=0.7\linewidth]{images/comparative-1-distilbert-sst2-imdb_learning_curve_worst_slice.pdf}
\caption{Worst-slice constrained accuracy over optimization steps for the Bonferroni baseline run (distilbert-labeled configuration); higher values indicate better performance.}
\label{fig:bonferroni_distilbert_learning_curve_worst_slice}
\end{figure}

\begin{figure}[H]
\centering
\includegraphics[width=0.7\linewidth]{images/comparative-1-flan-t5-small-imdb_confusion_imdb.pdf}
\caption{Confusion matrix on IMDb for the Bonferroni baseline run (flan-t5-small configuration); higher values on the diagonal indicate better performance.}
\label{fig:bonferroni_flan_confusion_imdb}
\end{figure}

\begin{figure}[H]
\centering
\includegraphics[width=0.7\linewidth]{images/comparative-1-flan-t5-small-imdb_confusion_sst2.pdf}
\caption{Confusion matrix on SST-2 for the Bonferroni baseline run (flan-t5-small configuration); higher values on the diagonal indicate better performance.}
\label{fig:bonferroni_flan_confusion_sst2}
\end{figure}

\begin{figure}[H]
\centering
\includegraphics[width=0.7\linewidth]{images/comparative-1-flan-t5-small-imdb_confusion_yelp.pdf}
\caption{Confusion matrix on Yelp for the Bonferroni baseline run (flan-t5-small configuration); higher values on the diagonal indicate better performance.}
\label{fig:bonferroni_flan_confusion_yelp}
\end{figure}

\begin{figure}[H]
\centering
\includegraphics[width=0.7\linewidth]{images/comparative-1-flan-t5-small-imdb_domain_accuracy.pdf}
\caption{Per-domain constrained accuracy summary for the Bonferroni baseline run (flan-t5-small configuration); higher values indicate better performance.}
\label{fig:bonferroni_flan_domain_accuracy}
\end{figure}

\begin{figure}[H]
\centering
\includegraphics[width=0.7\linewidth]{images/comparative-1-flan-t5-small-imdb_learning_curve_robust.pdf}
\caption{Robust constrained accuracy over optimization steps for the Bonferroni baseline run (flan-t5-small configuration); higher values indicate better performance.}
\label{fig:bonferroni_flan_learning_curve_robust}
\end{figure}

\begin{figure}[H]
\centering
\includegraphics[width=0.7\linewidth]{images/comparative-1-flan-t5-small-imdb_learning_curve_worst_slice.pdf}
\caption{Worst-slice constrained accuracy over optimization steps for the Bonferroni baseline run (flan-t5-small configuration); higher values indicate better performance.}
\label{fig:bonferroni_flan_learning_curve_worst_slice}
\end{figure}

\begin{figure}[H]
\centering
\includegraphics[width=0.7\linewidth]{images/proposed-distilbert-sst2-imdb_confusion_imdb.pdf}
\caption{Confusion matrix on IMDb for the proposed CFS-eSAFE run (distilbert-labeled configuration); higher values on the diagonal indicate better performance.}
\label{fig:proposed_distilbert_confusion_imdb}
\end{figure}

\begin{figure}[H]
\centering
\includegraphics[width=0.7\linewidth]{images/proposed-distilbert-sst2-imdb_confusion_sst2.pdf}
\caption{Confusion matrix on SST-2 for the proposed CFS-eSAFE run (distilbert-labeled configuration); higher values on the diagonal indicate better performance.}
\label{fig:proposed_distilbert_confusion_sst2}
\end{figure}

\begin{figure}[H]
\centering
\includegraphics[width=0.7\linewidth]{images/proposed-distilbert-sst2-imdb_confusion_yelp.pdf}
\caption{Confusion matrix on Yelp for the proposed CFS-eSAFE run (distilbert-labeled configuration); higher values on the diagonal indicate better performance.}
\label{fig:proposed_distilbert_confusion_yelp}
\end{figure}

\begin{figure}[H]
\centering
\includegraphics[width=0.7\linewidth]{images/proposed-distilbert-sst2-imdb_domain_accuracy.pdf}
\caption{Per-domain constrained accuracy summary for the proposed CFS-eSAFE run (distilbert-labeled configuration); higher values indicate better performance.}
\label{fig:proposed_distilbert_domain_accuracy}
\end{figure}

\begin{figure}[H]
\centering
\includegraphics[width=0.7\linewidth]{images/proposed-distilbert-sst2-imdb_learning_curve_robust.pdf}
\caption{Robust constrained accuracy over optimization steps for the proposed CFS-eSAFE run (distilbert-labeled configuration); higher values indicate better performance.}
\label{fig:proposed_distilbert_learning_curve_robust}
\end{figure}

\begin{figure}[H]
\centering
\includegraphics[width=0.7\linewidth]{images/proposed-distilbert-sst2-imdb_learning_curve_worst_slice.pdf}
\caption{Worst-slice constrained accuracy over optimization steps for the proposed CFS-eSAFE run (distilbert-labeled configuration); higher values indicate better performance.}
\label{fig:proposed_distilbert_learning_curve_worst_slice}
\end{figure}

\begin{figure}[H]
\centering
\includegraphics[width=0.7\linewidth]{images/proposed-flan-t5-small-imdb_confusion_imdb.pdf}
\caption{Confusion matrix on IMDb for the proposed CFS-eSAFE run (flan-t5-small configuration); higher values on the diagonal indicate better performance.}
\label{fig:proposed_flan_confusion_imdb}
\end{figure}

\begin{figure}[H]
\centering
\includegraphics[width=0.7\linewidth]{images/proposed-flan-t5-small-imdb_confusion_sst2.pdf}
\caption{Confusion matrix on SST-2 for the proposed CFS-eSAFE run (flan-t5-small configuration); higher values on the diagonal indicate better performance.}
\label{fig:proposed_flan_confusion_sst2}
\end{figure}

\begin{figure}[H]
\centering
\includegraphics[width=0.7\linewidth]{images/proposed-flan-t5-small-imdb_confusion_yelp.pdf}
\caption{Confusion matrix on Yelp for the proposed CFS-eSAFE run (flan-t5-small configuration); higher values on the diagonal indicate better performance.}
\label{fig:proposed_flan_confusion_yelp}
\end{figure}

\begin{figure}[H]
\centering
\includegraphics[width=0.7\linewidth]{images/proposed-flan-t5-small-imdb_domain_accuracy.pdf}
\caption{Per-domain constrained accuracy summary for the proposed CFS-eSAFE run (flan-t5-small configuration); higher values indicate better performance.}
\label{fig:proposed_flan_domain_accuracy}
\end{figure}

\begin{figure}[H]
\centering
\includegraphics[width=0.7\linewidth]{images/proposed-flan-t5-small-imdb_learning_curve_robust.pdf}
\caption{Robust constrained accuracy over optimization steps for the proposed CFS-eSAFE run (flan-t5-small configuration); higher values indicate better performance.}
\label{fig:proposed_flan_learning_curve_robust}
\end{figure}

\begin{figure}[H]
\centering
\includegraphics[width=0.7\linewidth]{images/proposed-flan-t5-small-imdb_learning_curve_worst_slice.pdf}
\caption{Worst-slice constrained accuracy over optimization steps for the proposed CFS-eSAFE run (flan-t5-small configuration); higher values indicate better performance.}
\label{fig:proposed_flan_learning_curve_worst_slice}
\end{figure}

\begin{figure}[H]
\centering
\includegraphics[width=0.7\linewidth]{images/comparison_accuracy_bar_chart.pdf}
\caption{Cross-run comparison of robust constrained accuracy; higher values indicate better performance.}
\label{fig:comparison_accuracy_bar_chart}
\end{figure}

\begin{figure}[H]
\centering
\includegraphics[width=0.7\linewidth]{images/comparison_accuracy_box_plot.pdf}
\caption{Cross-run seed-level distribution of robust constrained accuracy; higher values indicate better performance.}
\label{fig:comparison_accuracy_box_plot}
\end{figure}

\begin{figure}[H]
\centering
\includegraphics[width=0.7\linewidth]{images/comparison_metrics_table.pdf}
\caption{Cross-run metrics summary table visualization; higher is better for robust and worst-slice constrained accuracy and for accepted steps, while lower is better for OOD regression rate.}
\label{fig:comparison_metrics_table}
\end{figure}

\begin{figure}[H]
\centering
\includegraphics[width=0.7\linewidth]{images/comparison_significance_test.pdf}
\caption{Proposed versus baseline significance test summary plot; higher robust constrained accuracy would indicate better performance, though all observed values are $0.0$ in the executed runs.}
\label{fig:comparison_significance_test}
\end{figure}

%%%%%%%%%%%%%%%%%%%%%%%%%%%%%%%%%%%%%%%%%%
\section{Discussion}



%%%%%%%%%%%%%%%%%%%%%%%%%%%%%%%%%%%%%%%%%%
\section{Conclusions}

CFS-eSAFE is a training-free prompt optimization framework intended to provide an anytime-valid, deployment-oriented non-regression contract not only on an overall audit set but also on an adaptively discovered ordered list of stress slices. The method composes cross-fitted slice discovery (to avoid selection leakage), paired-difference evaluation, anytime-valid mixture e-values (to support optional stopping and repeated candidate testing), fixed-sequence (gatekeeping) slice certification (to avoid Bonferroni-style power collapse when many slices are audited), and alpha-wealth (to budget risk across an open-ended adaptive search).

The executed experiments do not empirically validate these intended advantages. Across all reported runs, constrained accuracy is $0.0$ on SST-2, Yelp, and IMDb; worst-slice constrained accuracy is $0.0$; and the optimizer evaluates zero candidates and accepts zero prompt updates. Confusion matrices are identical across methods and runs. The presence of some nonzero target matches in the confusion matrices alongside zero constrained accuracy indicates that semantic similarity and/or fluency constraints fail almost universally under the selected thresholds.

The main implication is methodological: before safety mechanisms can be compared, the underlying task instantiation must be non-degenerate. Two prerequisites emerge directly from the run summaries and logged outcomes: (i) calibrate $\tau_{\mathrm{similarity}}$ and the fluency threshold ($\mathrm{nll}_{\max}$, equivalently $\mathrm{ppl}_{\max}$) so that constrained success is achievable at a nontrivial rate for the base prompt; and (ii) ensure the candidate proposal and filtering stage reliably generates and evaluates candidate prompts so that gating, slice discovery, and regression certificates are exercised.

Once these prerequisites are met, the core hypothesis becomes testable with the same framework and code structure: whether fixed-sequence e-value gatekeeping can certify many adaptively discovered slices with higher acceptance power than Bonferroni while preserving interpretable regression certificates and end-to-end validity under adaptive, optionally stopped prompt search.

%%%%%%%%%%%%%%%%%%%%%%%%%%%%%%%%%%%%%%%%%%
\vspace{6pt}

%%%%%%%%%%%%%%%%%%%%%%%%%%%%%%%%%%%%%%%%%%
\authorcontributions{For research articles with several authors, a short paragraph specifying their individual contributions must be provided. The following statements should be used ``Conceptualization, X.X. and Y.Y.; methodology, X.X.; software, X.X.; validation, X.X., Y.Y. and Z.Z.; formal analysis, X.X.; investigation, X.X.; resources, X.X.; data curation, X.X.; writing---original draft preparation, X.X.; writing---review and editing, X.X.; visualization, X.X.; supervision, X.X.; project administration, X.X.; funding acquisition, Y.Y. All authors have read and agreed to the published version of the manuscript.'' Please turn to the  \href{http://img.mdpi.org/data/contributor-role-instruction.pdf}{CRediT taxonomy} for the term explanation. Authorship must be limited to those who have contributed substantially to the work~reported.}

\funding{This research received no external funding.}

\dataavailability{All resources used in this study are openly available at }

\acknowledgments{In this study, we automatically carried out a series of research processes—from hypothesis formulation to paper writing—using generative AI.}

\conflictsofinterest{The authors declare no conflicts of interest.}

%%%%%%%%%%%%%%%%%%%%%%%%%%%%%%%%%%%%%%%%%%
%\isPreprints{}{ % This command is only used for ``preprints''.
\begin{adjustwidth}{-\extralength}{0cm}
%} % If the paper is ``preprints'', please uncomment this parenthesis.
%\printendnotes[custom] % Un-comment to print a list of endnotes

\bibliographystyle{plainnat}
\bibliography{references}

\PublishersNote{}
%\isPreprints{}{ % This command is only used for ``preprints''.
\end{adjustwidth}
%} % If the paper is ``preprints'', please uncomment this parenthesis.
\end{document}
